\documentclass[12pt, a4paper,oneside]{book}
\usepackage[UTF8]{ctex} 

% Math and symbols
\usepackage{amsmath, amssymb, amsfonts, bm}
\usepackage{graphicx}
\usepackage{geometry}
\usepackage{titlesec}
\usepackage{epigraph}  % 加在导言区
\usepackage{titleps}
\usepackage{mathrsfs}
\usepackage{lipsum}




% 链接
\usepackage[colorlinks,linkcolor=blue,citecolor=blue,urlcolor=blue]{hyperref}


\geometry{margin=2.5cm}

% Colored boxes
\usepackage[many]{tcolorbox}
\tcbuselibrary{breakable}

\title{Linear Algebra}
\author{Eric Clapton\thanks{inspired by Pink Floyd}}
\date{\today}

% ---------- Box Environments ----------
% Theorem
\newtcolorbox[auto counter, number within=section]{theorem}[2][]{%
  colback=blue!5,
  colframe=blue!60!black,
  colbacktitle=blue!60!black,
  coltitle=white,
  fonttitle=\bfseries,
  title=Theorem~\thetcbcounter: #2,
  enhanced,
  breakable,
  attach boxed title to top left={yshift=-2mm, xshift=0.5cm},
  #1
}

% Definition
\newtcolorbox[auto counter, number within=section]{definition}[2][]{%
  colback=green!5,
  colframe=green!55!black,
  colbacktitle=green!55!black,
  coltitle=white,
  fonttitle=\bfseries,
  title=Definition~\thetcbcounter: #2,
  enhanced,
  breakable,
  attach boxed title to top left={yshift=-2mm, xshift=0.5cm},
  #1
}

% Lemma
\newtcolorbox[auto counter, number within=section]{lemma}[2][]{%
  colback=cyan!5,
  colframe=cyan!60!black,
  colbacktitle=cyan!60!black,
  coltitle=white,
  fonttitle=\bfseries,
  title=Lemma~\thetcbcounter: #2,
  enhanced,
  breakable,
  attach boxed title to top left={yshift=-2mm, xshift=0.5cm},
  #1
}

% Proposition
\newtcolorbox[auto counter, number within=section]{proposition}[2][]{%
  colback=orange!5,
  colframe=orange!75!black,
  colbacktitle=orange!75!black,
  coltitle=white,
  fonttitle=\bfseries,
  title=Proposition~\thetcbcounter: #2,
  enhanced,
  breakable,
  attach boxed title to top left={yshift=-2mm, xshift=0.5cm},
  #1
}

% Corollary
\newtcolorbox[auto counter, number within=section]{corollary}[2][]{%
  colback=purple!5,
  colframe=purple!60!black,
  colbacktitle=purple!60!black,
  coltitle=white,
  fonttitle=\bfseries,
  title=Corollary~\thetcbcounter: #2,
  enhanced,
  breakable,
  attach boxed title to top left={yshift=-2mm, xshift=0.5cm},
  #1
}

% Example
\newtcolorbox[auto counter, number within=section]{example}[2][]{%
  colback=yellow!5,
  colframe=yellow!60!black,
  colbacktitle=yellow!60!black,
  coltitle=black,
  fonttitle=\bfseries,
  title=Example~\thetcbcounter: #2,
  enhanced,
  breakable,
  attach boxed title to top left={yshift=-2mm, xshift=0.5cm},
  #1
}

% Proof
\newtcolorbox{proof}[1][]{%
  colback=pink!10,
  colframe=pink!80!black,
  colbacktitle=pink!70!black,
  coltitle=white,
  fonttitle=\bfseries,
  title=Proof,
  enhanced,
  breakable,
  attach boxed title to top left={yshift=-2mm, xshift=0.5cm},
  before upper={},
  after upper={\hfill$\square$},
  #1
}

% solution
\newtcolorbox[auto counter, number within=section]{solution}[2][]{%
  colback=gray!10,
  colframe=gray!70!black,
  colbacktitle=gray!70!black,
  coltitle=white,
  fonttitle=\bfseries,
  title=Solution~\thetcbcounter: #2,
  enhanced,
  breakable,
  attach boxed title to top left={yshift=-2mm, xshift=0.5cm},
  #1
}

% Conclusion
\newtcolorbox[auto counter, number within=section]{conclusion}[2][]{%
  colback=red!5,
  colframe=red!60!black,
  colbacktitle=red!60!black,
  coltitle=white,
  fonttitle=\bfseries,
  title=Conclusion~\thetcbcounter: #2,
  enhanced,
  breakable,
  attach boxed title to top left={yshift=-2mm, xshift=0.5cm},
  #1
}

% Exercise
\newtcolorbox[auto counter, number within=section]{exercise}[2][]{%
  colback=blue!5,
  colframe=blue!60!black,
  colbacktitle=blue!60!black,
  coltitle=white,
  fonttitle=\bfseries,
  title=Exercise~\thetcbcounter: #2,
  enhanced,
  breakable,
  attach boxed title to top left={yshift=-2mm, xshift=0.5cm},
  #1
}

% Remark
\newtcolorbox[auto counter, number within=section]{remark}[2][]{%
  colback=purple!5,
  colframe=purple!60!black,
  colbacktitle=purple!60!black,
  coltitle=white,
  fonttitle=\bfseries,
  title=Remark~\thetcbcounter: #2,
  enhanced,
  breakable,
  attach boxed title to top left={yshift=-2mm, xshift=0.5cm},
  #1
}

% Axiom
\newtcolorbox[auto counter, number within=section]{axiom}[2][]{%
  colback=magenta!5,
  colframe=magenta!60!black,
  colbacktitle=magenta!60!black,
  coltitle=white,
  fonttitle=\bfseries,
  title=Axiom~\thetcbcounter: #2,
  enhanced,
  breakable,
  attach boxed title to top left={yshift=-2mm, xshift=0.5cm},
  #1
}



\begin{document}


\begin{titlepage}
  \centering
  \includegraphics[width=0.6\textwidth]{cover.jpg}\\
  \vspace{2cm}
  {\Huge\bfseries Notes for Measure theory\par}
  \vspace{1.5cm}
  {\Large Based on \textit{Measure Theory} by Donald.Cohn \par}
  \vspace{2cm}
  {\Large\itshape Eric Clapton\par}
  \vfill
  {\large  \today}
\end{titlepage}

\chapter*{Preface}
\addcontentsline{toc}{chapter}{Preface}

The cover is from the album \textbf{\textit{Orange Moon}} by \textit{Khalil Fong}. The note 
is based on \textit{Measure Theory} by Donald.Cohn, \textit{Set Theory} by Jech.

\vspace{1cm}
\begin{flushright}
    --- Eric
\end{flushright}

\tableofcontents




\chapter{Introduction}

\setlength{\unitlength}{1pt}
\setlength{\epigraphwidth}{9cm}

\epigraph{“也许是天气,也许是运气,也许是因为有人不放弃。”}{\textit{— 方大同, 《因为你》}}


\section{Riemann integral -- Darboux's Definition}

 Let $[a,b]$ be a closed bounded interval. A \textbf{partition} of $[a,b]$ 
 is a \textbf{finite sequence} $\{a_i\}_{i=0}^k$ of real numbers such that 
 \begin{equation}
 a=a_0<a_1<\cdots <a_k=b
    \end{equation}
    we call the value $a_i$ the \textit{division points} of the partition. 
    the partition is denoted by a symbol $\mathscr{P}$. \par 
    Suppose that $f$ is a bounded real-valued function on $[a,b]$ 
    and that $\mathscr{P}$ is a partition of $[a,b]$, say with division points 
    $\{a_i\}_{i=0}^k$. We define $m_i$ and $M_i$ by $m_i=inf\{{f(x):x\in [a_{i-1},a_i]}\}$ 
    and $M_i=sup\{f(x):x\in [a_{i-1},a_i]\}$. Then the \textit{lower sum} is define 
    to be $\sum_{i=1}^k m_i(a_i-a_{i-1})$ and the \textit{upper sum} is defined 
    to be $\sum_{i=1}^k M_i(a_i-a_{i-1})$.\par
    Since $f$ is bounded, so there are $m$ and $M$ such that $m\leq f \leq M$. So the lower sum satisfies
    \begin{equation}
        \sum_{i=1}^k m_i(a_i-a_{i-1})\leq \sum_{i=1}^k M_i(a_i-a_{i-1})=M(b-a)
    \end{equation}
    So the set of lower sums of $f$ which denoted by $\underline{\int}_{a}^{b}$ has a supremum. Similarly, the upper 
    sums of $f$ which denoted by $\overline{\int}_{a}^{b}$ has a infimum. If $\underline{\int}_{a}^{b}=\overline{\int}_{a}^{b}$, which means that
    when the number of division points tends to infinity, the lower sum equals the upper sum, then $f$ is said 
    to be \textit{Riemann integrable} on $[a,b]$.
    
    \section{Riemann Integral -- Riemann's Definition}
    A \textit{tagged partition} of an interval $[a,b]$ is a partition $\{a_i\}_{i=0}{k}$ of $[a,b]$, 
    together with a sequence $\{x_i\}_{i=1}{k}$ of \textit{tags} 
    such that $a_{i-1}\leq x_i \leq a_i$ holds for $i=1,\cdots,k$. \par
    The \textit{mesh} $\|\mathscr{P}\|$ of a partition $\mathscr{P}$ is defined by $\|\mathscr{P}\|=max_i(a_i-a_{i-1})$, 
    THe mesh of a partition is the length of the longest of its subintervals. \par
    The Riemann sum $\mathscr{R}(f,\mathscr{P})$ is defined by 
    \begin{equation}
      \mathscr{R}(f,\mathscr{P})=\sum_{i=1}^{k}f(x_i)(a_i-a_{i-1})
    \end{equation}
    If there is a number $L$ such that 
    \begin{equation}
      \lim_{\mathscr{P}}\mathscr{R}(f,\mathscr{P})=L
    \end{equation}
    The limit is taken as the \textit{mesh} of $\mathscr{P}$ approachs 0.\par
    Darboux's and Riemann's definition are equivalent.Every continuous function on 
    $[a,b]$ is Riemann integrable.

    \begin{theorem}{The Fundamental Theorem of Calculus}
      Suppose that $f:[a,b]\rightarrow \mathbb{R}$ is continuous and that 
      $F:[a,b]\rightarrow \mathbb{R}$ is defined by $F(x)=\int_{a}^{x}f(t)dt$. 
      Then $F$ is differentiable at each $x$ in $[a,b]$ and its derivative is given by
      $F'(x)=f(x)$.
    \end{theorem}

    \section{From Riemann to Lebesgue}
    In many situations, it's necessary to reverse the order of taking limits and 
    evaluating integrals, which means that
    \begin{equation}
      \int_{a}^{b} \lim_{n}f_n(x)dx=\lim_{n}\int_{a}^{b}f_n(x)dx
    \end{equation}
    We need a theorem about that. 
    \begin{theorem}{When can we reverse the order of limit and integral}
    
      Suppose that $\{f_n(x)\}$ is a sequence of integrable functions on the 
      interval $[a,b]$ and that $f$ is a function that $\{f_n\}$ converges to $f$ 
      in a suitable \footnote{Of course we should figure out what "suitable" means}
      Then $f$ is integrable and 
      \begin{equation}
        \int_{a}^{b}f(x)dx=\lim_{n}\int_{a}^{b}f_n(x)dx 
      \end{equation} 
      \begin{equation}
        f(x)=\lim_{n}f_n(x)
      \end{equation}
      which means that$\int_{a}^{b}\lim_{n}f_n(x)=\lim_{n}\int_{a}^{b}f_n(x)dx$ 
    \end{theorem}
    If the converge is \textit{"converge uniformly"}\footnote{We will figure it out later}, we would say the theorem is valid 
    for the Riemann integral.\par 
    But if it is only pointwise\footnote{$
    \{f_n\}$ converges pointwise to $f$ on $[a,b]$ if $\lim_{n}f_n(x)=f(x)$ for each $x$ in $[a,b]$
    } converge. Then the theorem may fail. 

    \begin{example}{a example that makes Theorem 1.3.1 fail}
      For each positive integer $n$ let $f_n$ be a piecewise linear function分段线性函数
      on $[0,1]$ whose graph is made up of three line segments, connecting the points 
      $(0,0),(\frac{1}{2n},2n),(\frac{1}{n},0)$ and $(1,0)$. Then for each n the triangle formed by the
      graph of $f_n$ and the $x$-axis has area 1. So for every $f_n$ we have $\int_{0}^{1}f_n(x)=1$.
      but $\int_{0}^{1} \lim_{n}f_n(x)dx$ doesn't exist in Riemann integrals, which means that 
      $\int_{a}^{b}\lim_{n}f(x)dx \neq \lim_{n} \int_{a}^{b}f_n(x) $
      So Theorem 1.3.1 fails in this example. 
    \end{example}

    \begin{remark}{Explaination}
      But why did it fail? It's because $f_n(x)$ is not bounded uniformly, which means that there
      exist a $M$ and $|f_n(x)|\leq M$ holds for all $n$ and $x$.
    \end{remark}

    Next we look at an example in which the $f_n$ are uniformly bounded but the theorem still fails. 
   
    \begin{example}{A bounded function but not Riemann integrable}
      Since rational numbers are \textit{countable}.


      for each $n$ we define a function $f_n(x):[0,1]\rightarrow \mathbb{R}$ by
      \begin{equation}
        f_n(x) = 
        \begin{cases}
          1 & \text{if } x \in \mathbb{Q} \cap [0,1] \\
          0 & \text{if } x \in (\mathbb{R} \setminus \mathbb{Q}) \cap [0,1]
        \end{cases}
      \end{equation}
      Since the rationals are \textbf{dense} in $[0,1]$. 
      So the $\underline{\int}_{0}^{1}=0$ and $\overline{\int}_{0}^{1}=1$. 
      So $f$ is not Riemann integrable. 
      

    \end{example}

    \begin{example}{The difficulty to make Theorem 1.3.1 valid}
      The difficulty in the previous example comes from the fact 
      that the $f_n$ fail to be continuous.But you can also produce a sequence $\{f_n(x)\}$ 
      such that 
      \begin{enumerate}
      \item each $f_n(x)$ is continuous, 
      \item $0\leq f_n(x)\leq 1$ holds for each $n$ and $x$
      \item $\{f_n(x)\}$ converges pointwise to a function that is not Riemann integrable.
      \end{enumerate}
    \end{example}

    Lebesgue showed that Theorem 1.3.1, when formulated in terms of his new integral,
  holds for pointwise convergence of the sequence ${f_n}$, subject only to some rather
  natural boundedness conditions on that sequence. \par
    The definition of Riemann Integral deals with partition of the interval $[a,b]$, 
  which is the domain of $f$. But the Lebesgue Integral deals with the partition of the range of $f$. \par
    Suppose we have $c$ to be a positive number and $0\leq f(x)\leq c$ holds for $x\in [a,b]$.
  Suppose that $\mathscr{p}$ is a partition of $[0,c]$, and we have
  \begin{equation}
    A_i={x\in[a,b]:f(x)\in [a_{i-1},a_i)}
  \end{equation}
  the set $A_i$ can be empty, unions of finite collection of subintervals, or even more complicated sets. 
  the sum $s(f,\mathscr{P})$ is given by
  \begin{equation}
    s(f,\mathscr{P})=\sum_{i=1}^{k}a_{i-1}meas(A_i)
  \end{equation}

  meas($A_i$)is the \textbf{size} of the \textbf{set $A_i$}. 
  How to define the size of set is the problem we need to solve. 

  \chapter{Measures}
  \setlength{\unitlength}{1pt}
  \setlength{\epigraphwidth}{9cm}

  \epigraph{“人间的青草地需要浇水,内心的花园就不会枯萎。”}{\textit{— 方大同, 《每个人都会》}}

  \section{Algebras and Sigma-Algebras}
  \begin{definition}{Algebra}
    Let $X$ is an arbitrary set.A collection $\mathscr{A}$ of subsets of $X$ is an 
    \textit{algebra} on $X$ if 
    \begin{enumerate}
      \item  $X\in \mathscr{A}$,
      \item for each set $A$ that belongs to $\mathscr{A}$, the set $A^c$ belongs to $\mathscr{A}$. 
      \item for each finite sequence $A_1\cdots,A_n$ of sets that belongs to $\mathscr{A}$, the set $\cup_{i=1}^n A_i$ 
      belongs to $\mathscr{A}$, and 
      \item for each finite sequence $A_1\cdots,A_n$ of sets that belongs to $\mathscr{A}$, the set $\cap_{i=1}^n A_i$ 
      belongs to $\mathscr{A}$. 
    \end{enumerate}
  \end{definition}
   From the definition we can find out that $\mathscr{A}$ is closed under complementation, finite unions有限并集 and finite intersections有限交集 
   and condition(3) is equal to condition(4) under condition(2), because 
   $\cap_{i=1}^nA_i=(\cup_{i=1}^nA_i^c)^c$

   \begin{definition}{$\sigma$-Algebra}
    Let $X$ is an arbitrary set.A collection $\mathscr{A}$ of subsets of $X$ is an 
    \textit{$\sigma$-algebra} on $X$ if
    \begin{enumerate}
      \item  $X\in \mathscr{A}$,
      \item for each set $A$ that belongs to $\mathscr{A}$, the set $A^c$ belongs to $\mathscr{A}$. 
      \item for each countable infinite sequence $\{A_i\}$ of sets that belongs to $\mathscr{A}$, the set $\cup_{i=1}^{\infty} A_i$ 
      belongs to $\mathscr{A}$, and 
      \item for each countable infinite sequence $\{A_i\}$ of sets that belongs to $\mathscr{A}$, the set $\cap_{i=1}^{\infty} A_i$ 
      belongs to $\mathscr{A}$. 
    \end{enumerate}
   \end{definition}
   From the definition we can find out that $\mathscr{A}$ is closed under complementation, countable unions可数并集 and countable intersections可数交集 
   Note that, as in the case of algebras, we could
have used only conditions (1), (2), and (3), or only conditions (1), (2), and (4), in our
definition. 
   \begin{remark}{Each $\sigma$-Algebra on $X$ is an algebra on $X$}
    the union of the finite sequence $A_1,A_2,\cdots,A_n$ is the same as the 
    union of the infinite sequence $A_1,A_2,\cdots,A_n,A_n,A_n,\cdots$
   \end{remark}

   Thus in the definitions of algebras and $\sigma$-algebras given above, we can replace condition (1) with the
requirement that $\phi$ be a member of $\mathscr{A}$ . Furthermore, if $\mathscr{A}$ is a family of subsets of
$X$ that is nonempty, closed under complementation, and closed under the formation
of finite or countable unions, then $\mathscr{A}$ must contain $X$: if the set $A$ belongs to $\mathscr{A}$, then
$X$, since it is the union of $A$ and $A^c$, must also belong to $\mathscr{A}$ . Thus in our definitions
of algebras and $\sigma$-algebras, we can replace condition (1) with the requirement that
$\mathscr{A}$ be nonempty.\par
   If $\mathscr{A}$ is a $\sigma$-algebra on $X$, we call a subset of $X$ $\mathscr{A}-Measurable$
if it belongs to $\mathscr{A}$.
\begin{example}{Some example of Families of sets}

  \begin{enumerate}
    
    \item Let $X$ be a set, and let $\mathscr{A}$ be the collection of all 
    subsets of $X$. Then $\mathscr{A}$ is a $\sigma$-algebra of $X$.
    
    \item Let $X$ be a infinite set, and let $\mathscr{A}$ be the collection of all 
    subsets $A$ of $X$ such that either $A$ or $A^c$ is finite. Then $\mathscr{A}$ is an 
    algebra on $X$ but is not closed under countable unions; hence it's not a $\sigma $-algebra.
    
    \item Let $X$ be a set, and let $\mathscr{A}$ be the collection of all subsets $A$ of $X$ 
    such that $X$ such that either $A$ or $A^c$ is countable. Then $\mathscr{A}$ is a $\sigma$-algebra.         
  
    \item Let $\mathscr{A}$ be the collection of all subsets of $\mathbb{R}$ that are unions of 
    finitely many intervals of the form $(a,b]$, $(a,+\infty)$ or $(-\infty,b]$. The $\mathscr{A}$ is an algebra 
    but is not an $\sigma$-algebra. Suppose we have $A_n=(0,2-\frac{1}{n}]$, for every certain n, $A_n\in \mathscr{A}$, 
    but $\cup_{i=1}^\infty A_i =(0,2)\notin \mathscr{A}$. So it's not an $\sigma$-algebra.             
  
  \end{enumerate}

\end{example}
    
  Now let's consider way of constructing $\sigma$-algebra.
  \begin{proposition}{constructing $\sigma$-algebra}
    Let $X$ be a set. Then the intersection of an arbitrary nonempty collection of $\sigma$-algebras 
    on $X$ is a $\sigma$-algebra on $X$.                 
  \end{proposition}

  \begin{proof}
    This proof is easy so I just ignore it.         
  \end{proof}

  We should notice that the union of a family of $\sigma$-algebra may not be a $\sigma$-algebra. From the proof we can
  see that the key is that the property of intersection makes the transition of sets possible, but union may not succeed in doing this.
  
  \begin{corollary}{$\sigma$-algebra on $X$ includes a family subsets of $X$ }
    Let $X$ be a set, and let $\mathscr{F}$ be a family of subsets of $X$. Then there 
    is a smallest $\sigma$-algebra on $X$ that includes $\mathscr{F}$.      
    
  \end{corollary}

  \begin{remark}{smallest $\sigma$-algebra}
    When we say the \textit{smallest} $\sigma$-algebra(denoted by $\mathscr{A}$ ) that includes $\mathscr{F}$, we are saying that
    any $\sigma$-algebra that includes $\mathscr{F}$ also includes $\mathscr{A}$.
    The smallest $\sigma$-algebra is called the $\sigma$-algebra \textit{generated by $\mathscr{F}$} 
    and denoted by $\sigma(\mathscr{F})$.       

  \end{remark}

  \begin{proof}
    Let $\mathscr{L}$ be the collection of all the $\sigma$-algebra that include $\mathscr{F}$. 
    Then $\mathscr{L}$ is nonempty, since it contains the $\sigma$-algebra that consists of all subsets of 
    $X$. Based on the preceding proposition. The intersection of these $\sigma$-algebras which belongs to $\mathscr{L}$ is also a 
    $\sigma$-algebra. It includes $\mathscr{F}$ and is included in every $\sigma$-algebra in $\mathscr{L}$.        
  \end{proof}
   
   \begin{definition}{Borel $\sigma$-algebra on $\mathbb{R}^d$  }
    The \textit{Borel} $\sigma$-algebra on $\mathbb{R}^d$ is the $\sigma$-algebra on $\mathbb{R}^d$ 
    generated by the collection of open subsets of $\mathbb{R}^d$. 
    It is denoted by $\mathscr{B}(\mathbb{R}^d)$. The \textit{Borel subsets} of $\mathbb{R}^d$ 
    are those that belong to $\mathscr{B}(\mathbb{R}^d)$.          
   \end{definition}

   \begin{proposition}{generating Borel $\sigma$-algebra}
    The $\sigma$-algebra $\mathscr{B}(\mathbb{R})$ of Borel subsets of $\mathbb{R}$ is generated 
    by each of the following collections of sets:
    \begin{enumerate}
      \item the collection of all closed subsets of $\mathbb{R}$.
      \item the collection of all subintervals of $\mathbb{R}$ of the form $(-\infty,b]$.
      \item the collection of all subintervals of $\mathbb{R}$ of the form $(a,b]$.
    \end{enumerate}
        
   \end{proposition}
   
  
   \begin{proof}
    Let $\mathscr{B}_1,\mathscr{B}_2$,and $\mathscr{B}_3$ be the $\sigma$-algebras 
    generated by the collections of sets in parts (1),(2), and (3) of the proposition. 
    We will show that $\mathscr{B}(\mathbb{R})\supseteq\mathscr{B}_1\supseteq\mathscr{B}_2\supseteq\mathscr{B}_3$ and then that $\mathscr{B}_3\supseteq\mathscr{B}(\mathbb{R});$ 
    this will establish the proposition. Since $\mathscr{B}(\mathbb{R})$ includes the family of open subsets of $\mathbb{R}$ and is closed under complementation, it includes the family of closed subsets of $\mathbb{R};$ thus it includes the $\sigma$-algebra generated by the closed subsets of $\mathbb{R}$, namely $\mathscr{B}_1.$ The sets of the form $(-\infty,b]$ are closed and so belong to $\mathscr{B}_1;$ consequently $\mathscr{B}_1\supseteq\mathscr{B}_2.$ Since $(a,b]=(-\infty,b]\cap(-\infty,a]^c$,each set of the form $(a,b]$ belongs to $\mathscr{B}_2;$ thus $\mathscr{B}_2\supseteq\mathscr{B}_3.$ Finally, note that each open subinterval of $\mathbb{R}$ is the union of a sequence of sets of the form $(a,b]$ and that each open subset of $\mathbb{R}$ is the union of a sequence of open intervals (this proof needs the dense property of rational numbers). 
    Thus each open subset of $\mathbb{R}$ belongs to $\mathscr{B}_3$, and so $\mathscr{B}_3\supseteq\mathscr{B}(\mathbb{R}).$
   \end{proof}

   The Borel $\sigma$-algebra $\mathscr{B}(\mathbb{R})$ contains virtually every subset of 
   $\mathbb{R}$ that is interest of analysis, and it's small enough that it can be dealt 
   with in a fairly constructive manner.\par   
   
   Let $\mathscr{G}$ be the family of all open subsets of $\mathbb{R}^d$, and let $\mathscr{F}$ be the family of all 
   closed subsets of $\mathbb{R}^d$. Let $\mathscr{G}_\delta$ be the collection of all intersections of 
   sequence of sets in $\mathscr{G}$. Let $\mathscr{F}_\sigma$ be the collection of all intersections of 
   sequence of sets in $\mathscr{F}$. Sets in $\mathscr{G}_\delta$ are often called $G_\delta 's$, sets in $\mathscr{F}\sigma$ 
   are called $F_\sigma 's$.
   
   \begin{proposition}{Each closed subset of $\mathbb{R}^d$ is a $G_\delta$}
    \begin{proof}
      suppose $F$ is a closed set in $\mathbb{R}^d$, construct open a sequence $\{U_n\}$ 
      of open subsets of $\mathbb{R}^d$ such that $F=\cap_n U_n$. We define $U_n$ by:
      $$ U_n = \{x\in \mathbb{R}^d: \|x-y\|< 1/n \  \text{for some} \  y \  \text{in} \ F\} $$      
       
    
    \end{proof}
    
   \end{proposition}
   
  \begin{proposition}{Each open subset of $\mathbb{R}^d$ is an $\mathscr{F}_\sigma$}
    \begin{proof}
      If $U$ is an open set, then $U^c$ is a closed set, so $U^c$ is a $G_\delta$. 
      Therefore there is a sequence $\{U_n\}$ of open sets such that $U^c=\cap_n U_n$. 
      Based on De Morgan's Law, $U$ is an $F_\sigma$.       
    \end{proof}
  \end{proposition}
  
  A sequence $\{A_i\}$ of sets is called \textit{increasing} if $A_i\subseteq A_{i+1}$ holds 
  for each $i$ and \textit{decreasing} if $A_{i+1}\subseteq A_i$ holds for each $i$.
  
  \begin{proposition}{When can algebra be $\sigma$-algebra?  }
    Let $X$ be a set, and let $\mathscr{A}$ be an algebra on $X$. Then 
    $\mathscr{A}$ is a $\sigma$-algebra if either:
    \begin{enumerate}
      \item $\mathscr{A}$ is closed under the formation of unions of increasing sequences of sets, or
      \item $\mathscr{A}$ is closed under the formation of intersections of decreasing sequences of sets.  
    \end{enumerate}
        \begin{proof}
          Since $\mathscr{A}$ is an algebra, Suppose $\{A_i\}$ is a sequence of sets that belong to $\mathscr{A}$. 
          For each $n$ let $B_n=\cup_{i=1}^n A_i$. The sequence $\{B_n\}$ is increasing, since $\mathscr{A}$ is an algebra. 
          each $B_n$ belongs to $\mathscr{A}$. The assumption(1) implies that $\cup_n B_n$ belongs to $\mathscr{A}$. 
          However $\cup_i A_i$ is equal to $\cup_n B_n$ and so belongs to $\mathscr{A}$. So $\mathscr{A}$ is an 
          $\sigma$-algebra.\par
          If $\{A_i\}$ is an increasing sequence of sets that belong to $\mathscr{A}$, then $\{A_i^c\}$ is a decreasing 
          sequence of sets that belong to $\mathscr{A}$, so condition(2) implies $\cap_i A_i^c$ belongs to 
          $\mathscr{A}$. Since $\cup_i A_i = (\cap_i A_i^c)^c$, so $\cup_i A_i$ belongs to $\mathscr{A}$. So we 
          have $\mathscr{A}$ is a $\sigma$-algebra.          
        \end{proof}
        
  \end{proposition}
  
\section{Measures}
   
 


\chapter*{References}
\addcontentsline{toc}{chapter}{References}


\bibliographystyle{plain}
\bibliography{reference-marks}


\end{document}
