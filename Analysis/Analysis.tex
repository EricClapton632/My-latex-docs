%

\documentclass[12pt, a4paper, oneside]{book}

% 数学与符号支持

\usepackage{amsmath, amssymb, amsfonts, bm}

\usepackage{mathrsfs}

\usepackage{enumitem} 

% 图形

\usepackage{graphicx}

% 页边距

\usepackage{geometry}

\geometry{margin=2.5cm}

% 标题格式

\usepackage{titlesec}

% 超链接

\usepackage[colorlinks=true, linkcolor=blue, citecolor=blue, urlcolor=blue]{hyperref}

% 自动翻译引用前缀(cref)

\usepackage{cleveref}


% -------------------------


% 定义定理环境(统一编号) - AMS 推荐风格

% -------------------------
\usepackage{amsthm}


% 定理类:斜体(italic)正文

\usepackage{parskip}
\setlength{\parindent}{2em}
\setlength{\parskip}{1em}


\theoremstyle{plain}


\newtheorem{theorem}{Theorem}[section]

\newtheorem{lemma}[theorem]{Lemma}

\newtheorem{corollary}[theorem]{Corollary}

\newtheorem{proposition}[theorem]{Proposition}

% 定义类:正体(upright)正文

\theoremstyle{definition}

\newtheorem{definition}[theorem]{Definition}

\newtheorem{example}[theorem]{Example}

\newtheorem{remark}[theorem]{Remark}

\newtheorem{conclusion}[theorem]{Conclusion}

\newtheorem{solution}[theorem]{Solution}

% 备注类:标题斜体 + 正文正体

%\theoremstyle{remark}

% 证明环境默认存在(含自动小方块)

% -------------------------

% cleveref 中文化设置(可根据需要改为英文或保留)

% -------------------------

\crefname{theorem}{Theorem}{Theorems}

\crefname{lemma}{Lemma}{Lemmas}

\crefname{corollary}{Corollary}{Corollaries}

\crefname{proposition}{Proposition}{Propositions}

\crefname{definition}{Definition}{Definitions}

\crefname{example}{Example}{Examples}

\crefname{remark}{Remark}{Remarks}

\crefname{equation}{Equation}{Equations}

% -------------------------

% 正文开始

% -------------------------

\begin{document}

% -------------------------

% 封面

% -------------------------

\begin{titlepage}

\centering

\includegraphics[width=0.6\textwidth]{cover.jpg}\\

\vspace{2cm}

{\Huge\bfseries Logic\par}

\vspace{1.5cm}

{\Large introduction to symbol logic\par}

\vspace{2cm}

{\Large\itshape Eric Clapton\par}

\vfill

{\large \today}

\end{titlepage}



% -------------------------

% 前言

% -------------------------

\chapter*{Preface}

\addcontentsline{toc}{chapter}{Preface}



This is an example of a math document written in \LaTeX{} using \texttt{amsthm}, supporting unified numbering for theorems, definitions, and examples, with \texttt{cleveref} for auto-referencing.



\vspace{1cm}

\begin{flushright}

--- Eric

\end{flushright}

  

% -------------------------

% 目录

% -------------------------

\tableofcontents
\chapter{Foundation}
\section{Fundamentals of Logic}

\textbf{Statements} can be meaningfully claimed to be true or false. \par
If $E(x)$ is an expression which becomes a statement when $x$ is replaced by 
an object or class, then $E$ is a \textbf{property}.\par

We write $\exists! x\in X:E(x)$ when exactly one object $\in X$ has property $E$ exists.\par

Let $A$ and $B$ be statements. Then we can define a new \textbf{statement} implication :
$$ (A\rightarrow B):= (\neg A) \lor B $$

\section{Sets}

If $X \subseteq Y$ and $X \neq Y$, then $X$ is called \textbf{proper subset} of $Y$.
We denote this relationship by $X \subset Y$.   \par 

If $X$ is a set and $E$ is a property then $\{x\in X; E(x)\}$ is the subset 
of $X$ consisting of all elements $x$ of $X$ such that $E(x)$ is true. Then the set:
$$ \phi_X := \{x\in X ; x\neq x\}$$
is called the \textbf{empty subset} of $X$. \\

\begin{remark}
    Let $E$ be a property, then 
    $$ x\in \phi_X \rightarrow E(x) $$ is true for each $x \in X$   
\end{remark}

\section{The Power Set}
The power set of $X$ is the set consists of all the subsets of $X$. Sometimes it's written 
$2^X$.   

\section{Families of Sets}
Let $\mathsf{A}$ be a noempty set and for each $\alpha \in \mathsf{A}$, let $\mathsf{A}_\alpha $ 
be a set. Then $\{A_\alpha ;\alpha \in \mathsf{A}$ is called a \textbf{family of sets} and 
$\mathsf{A}$ is an \textbf{index set} for this family.  

\section{function}
A \textbf{function} \tetxbf{from} $X$ \textbf{to} $Y$ is a rule which, for each element
of $X$, specifies exactly one element of $Y$. The set $X$ is called the \textbf{domain}
of $f$ and $Y$ is called \textbf{codomain} of $f$. And 
$$ im(f):=\{y\in Y;\exists x\in X:y=f(x)\}$$
is called the \textbf{image} of $f$.

\begin{remark}
    we should notice that the image of $f$ is not equal to the codomain of $f$.  
\end{remark}


\end{document}
