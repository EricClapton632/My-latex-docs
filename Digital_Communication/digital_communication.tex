\documentclass[12pt,a4paper]{article}

% --- 编码与字体 ---
\usepackage[utf8]{inputenc}
\usepackage[T1]{fontenc}
\usepackage{lmodern}
\usepackage{microtype}            % 字体微调,提升排版质量

% --- 页面设置 ---
\usepackage[a4paper, margin=2.5cm]{geometry}
\linespread{1.25}

% --- 数学相关宏包 ---
\usepackage{amsfonts, amsmath, amssymb, amsthm, mathtools, mathrsfs, bm, mathdots}
\usepackage{siunitx}              % 单位与数字格式化
\sisetup{detect-all}

% --- 定理环境 ---
\usepackage{amsthm}
\newtheorem{definition}{Definition}[section]
\newtheorem{theorem}[definition]{Theorem}
\newtheorem{lemma}[definition]{Lemma}
\newtheorem{corollary}[definition]{Corollary}
\newtheorem{proposition}[definition]{Proposition}
\theoremstyle{remark}
\newtheorem{remark}[definition]{Remark}
\newtheorem{example}[definition]{Example}

% --- 表格与列表 ---
\usepackage{booktabs, tabularx, multirow}
\usepackage{enumitem}            % 自定义 itemize/enumerate 格式

% --- 图形与绘图 ---
\usepackage{graphicx}
\usepackage{tikz}
\usepackage{tkz-euclide}         % 欧几里得几何图形
\usepackage{pgfplots}            % 函数图像、绘图
\pgfplotsset{compat=1.18}
\usepackage[all]{xy}             % 绘制交换图(如范畴论)

% --- 其他工具宏包 ---
\usepackage{caption}             % 图表标题
\usepackage{pdflscape}           % 横向页面
\usepackage{slashed}             % 狄拉克符号:\slashed{p}
\usepackage[normalem]{ulem}      % 下划线、删除线等(不覆盖 \emph)
\usepackage{alltt}               % 可插入命令的等宽文本环境
\usepackage{imakeidx}            % 索引
\makeindex

% --- 页眉页脚与超链接 ---
\usepackage{fancyhdr}
\usepackage[colorlinks=true, linkcolor=blue, citecolor=blue, urlcolor=blue]{hyperref}
\pagestyle{fancy}
\fancyhf{}
\fancyhead[L]{\leftmark}
\fancyhead[R]{\thepage}

% --- 标题设置 ---
\usepackage{titlesec}
\titleformat{\section}{\normalfont\Large\bfseries}{\thesection.}{1em}{}
\titleformat{\subsection}{\normalfont\large\bfseries}{\thesubsection.}{0.5em}{}

% --- 封面信息 ---
\title{\textbf{Notes on Digital communication}}
\author{Eric Clapton \\ \small Xidian University}
\date{\today}

\begin{document}

\maketitle
\newpage
\tableofcontents
\newpage



% --- 正文示例 ---
\section{Preliminaries}

\begin{definition}
	Let $A, B$ be sets. We say $A \subseteq B$ if every $a \in A$ also satisfies $a \in B$.
\end{definition}

\begin{theorem}[Pythagorean Theorem]
	Let $a, b$ be the legs of a right triangle and $c$ the hypotenuse. Then:
	\[
		a^2 + b^2 = c^2.
	\]
\end{theorem}




\begin{proof}
	This follows from Euclidean geometry.
\end{proof}

\begin{figure}[h]
	\centering
	\begin{tikzpicture}
		\draw[->] (-0.2,0) -- (3.5,0) node[right] {$x$};
		\draw[->] (0,-1.5) -- (0,1.5) node[above] {$y$};
		\draw[domain=0:3.14, smooth, variable=\x, blue, thick] plot ({\x}, {sin(\x r)});
	\end{tikzpicture}
	\caption{A sine function}
\end{figure}

\begin{landscape}
	\begin{table}[h]
		\centering
		\caption{Sample Data in Landscape Mode}
		\begin{tabular}{cccc}
			\toprule
			$x$     & $f(x)$ & $\nabla f(x)$ & $\int f(x)\,dx$ \\
			\midrule
			0       & 0      & 1             & 0               \\
			$\pi/2$ & 1      & 0             & 1               \\
			\bottomrule
		\end{tabular}
	\end{table}
\end{landscape}

\begin{example}
	Let $\slashed{p} = \gamma^\mu p_\mu$. Then $\slashed{p}^2 = p^2$ in Minkowski space.
\end{example}

\index{Pythagorean Theorem}
\index{Sine function}
\newpage
\printindex


\end{document}

