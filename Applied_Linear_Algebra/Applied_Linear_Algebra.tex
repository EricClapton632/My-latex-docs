\documentclass[12pt, a4paper, oneside]{book}

% 数学与符号支持
\usepackage{amsmath, amssymb, amsfonts, bm}
\usepackage{mathrsfs}

% 图形
\usepackage{graphicx}

% 页边距
\usepackage{geometry}
\geometry{margin=2.5cm}

\usepackage{epigraph}


% 标题格式
\usepackage{titlesec}

% 超链接
\usepackage[colorlinks=true, linkcolor=blue, citecolor=blue, urlcolor=blue]{hyperref}

% 自动翻译引用前缀(cref)
\usepackage{cleveref}

% -------------------------
% 定义定理环境(统一编号) - AMS 推荐风格
% -------------------------
\usepackage{amsthm}

% 定理类:斜体(italic)正文
\theoremstyle{plain}
\newtheorem{theorem}{Theorem}[section]
\newtheorem{lemma}[theorem]{Lemma}
\newtheorem{corollary}[theorem]{Corollary}
\newtheorem{proposition}[theorem]{Proposition}

% 定义类:正体(upright)正文
\theoremstyle{definition}
\newtheorem{definition}[theorem]{Definition}
\newtheorem{example}[theorem]{Example}
\newtheorem{remark}[theorem]{Remark}
\newtheorem{conclusion}[theorem]{Conclusion}
\newtheorem{solution}[theorem]{Solution}

% 备注类:标题斜体 + 正文正体
%\theoremstyle{remark}



% 证明环境默认存在(含自动小方块)

% -------------------------
% cleveref 中文化设置(可根据需要改为英文或保留)
% -------------------------
\crefname{theorem}{Theorem}{Theorems}
\crefname{lemma}{Lemma}{Lemmas}
\crefname{corollary}{Corollary}{Corollaries}
\crefname{proposition}{Proposition}{Propositions}
\crefname{definition}{Definition}{Definitions}
\crefname{example}{Example}{Examples}
\crefname{remark}{Remark}{Remarks}
\crefname{equation}{Equation}{Equations}

% -------------------------
% 正文开始
% -------------------------

\begin{document}

\begin{titlepage}
  \centering
  \includegraphics[width=0.6\textwidth]{cover.jpg}\\
  \vspace{2cm}
  {\Huge\bfseries Notes for Linear Algebra\par}
  \vspace{1.5cm}
  %{\Large 副标题(可选)\par}
  %\vspace{2cm}
  {\Large\itshape Eric Clapton\par}
  \vfill
 % {\large 出版社名称\\ \today}
\end{titlepage}

\chapter*{Preface}
\addcontentsline{toc}{chapter}{Preface}

The cover is from the album \textbf{\textit{Dark side of the Moon}} by \textit{Pink Floyd}.

\vspace{1cm}
\begin{flushright}
    --- Eric
\end{flushright}

\tableofcontents

\chapter{Introduction}

\setlength{\unitlength}{1pt}
\setlength{\epigraphwidth}{12cm}

\epigraph{“But the greatest thing by far is to have a command of metaphor. This alone cannot be imparted by another; it is the mark of genius, for to make good metaphors implies an eye for resemblances.”}{\textit{— Aristotle, Poetics, 1459a7-10, tr. by S.H. Butcher.}}



\section{Why is Linear Algebra important? }


\begin{example}{matrix and information}
  $$\begin{array}
{cccccc}\mathrm{class}\setminus\mathrm{student} & 1 & 2 & 3 & 4 & 5 \\
1 & 1 & 0 & 0 & 1 & 1 \\
2 & 0 & 1 & 1 & 0 & 1 \\
3 & 1 & 0 & 1 & 0 & 1 \\
4 & 1 & 0 & 0 & 1 & 1
\end{array}$$
    we have 4 students and 5 classes,
    1 implies this student is in the class while 0 means not.
    $$A=
\begin{bmatrix}
1 & 0 & 0 & 1 & 1 \\
0 & 1 & 1 & 0 & 1 \\
1 & 0 & 1 & 0 & 1 \\
1 & 0 & 0 & 1 & 1
\end{bmatrix}.$$
$A_{ij}=1$ means the $i$th student is in the $j$th class, 
sum the row we can get the number of classes a student attends, 
sum the column we can get the number of students a class have. 
Our question is how can we calculate the number of students that take both
class 1 and 3? With a simple matrix we can easily get the answer. But if there are
thousands of students or class what way we should use? 
Look at this matrix multiplication. 
$$A^TA=
\begin{bmatrix}
1 & 0 & 1 & 1 \\
0 & 1 & 0 & 0 \\
0 & 1 & 1 & 0 \\
1 & 0 & 0 & 1 \\
1 & 1 & 1 & 1
\end{bmatrix}
\begin{bmatrix}
1 & 0 & 0 & 1 & 1 \\
0 & 1 & 1 & 0 & 1 \\
1 & 0 & 1 & 0 & 1 \\
1 & 0 & 0 & 1 & 1
\end{bmatrix}=
\begin{bmatrix}
3 & 0 & 1 & 2 & 3 \\
0 & 1 & 1 & 0 & 1 \\
1 & 1 & 2 & 0 & 2 \\
2 & 0 & 0 & 2 & 2 \\
3 & 1 & 2 & 2 & 4
\end{bmatrix}$$
This matrix is symmetric, Next, look at the diagonal entries: 3, 1, 2, 2, 4 - these give the total enrollments
in class 1, 2, 3, 4, and 5. Next, let's look at the entry in row 1, column
4. This is a 2. It's also how many students are enrolled in both classes 1
and 4. In fact, you can check that the $ij$th entry of $A^T A$ is the number
of students enrolled in both class $i$ and class $j$.
  \end{example}

  \begin{proof}
    if the $ij$th entry of $p\times q$ matrix $B$ is $B_{ij}$ 
    and the $ij$th entry of $q\times r$ matrix $C$ is $C_{ij}$
    then the $ij$th entry of $p\times r$ matrix $BC$ is $\sum_{\ell=1}^qB_{i\ell}C_{\ell j}$. 
    If A is a $m\times n$ matrix then $A^T$ is a $n\times m$ matrix, 
    so $(A^TA)_{ij}=\sum_{\ell=1}^m A_{i\ell}^T A_{\ell j}=
    \sum_{\ell=1}^m A_{\ell i} A_{\ell j}$. 
    $A_{\ell i} A_{\ell j}=1$ means that $A_{\ell i} = A_{\ell j}=1$. 
    So it means that the $\ell$th student is in the $i$ class and the 
    $\ell$th student is in the $j$th class. So we can check the $(A^TA)_{ij}$ 
    entry to see the number of students who attends both the $i$th and $j$th class. 

  \end{proof}

\section{Graphs and Matrices}
Encoding a graph into a matrix.

\begin{definition}
    A \textbf{Graph} is a pair $(V,E)$, where $V$ is a set of textbf{vertices} and 
    E is a subset of $V\times V$. That is: E is a subset of ordered pairs$(u,v)$ 
    where $u$ and $v$ are vertices. Elements of E are called \textbf{edges or links} 
    $(u,v)\in E$ means that $u$ and $v$ are connected. We require that $(u,u)\notin E$ 
    $(u,v)\in E $ if and only if $(v,u)\in E$. That means we consider undirected graphs.
    
    A \textbf{Path} is a finite sequence of vertices$(i_1,i_2,i_3, \dots ,i_k)$ 
    that $(i_j,i_{j+1})$ is an textbf{edge}. In other words, a path is a sequence of
    vertices in the graph such that each vertex is connected to the following
    vertex. The textbf{length}of a path is the number of edges that we travel along the path. 
    
    The \textbf{Degree} of a vertex $i$ is the number of vertices that are connected to $i$, 
    we write deg$i$ for the degree of $i$.  
    \end{definition}

    \begin{definition}
      The adjacency matrix $A$ of a graph$(V,E)$ has entries given by:
      $$A_{ij}=
\begin{cases}
1 & \mathrm{if}\left(i,j\right)\in E \\
0 & \mathrm{if}\left(i,j\right)\notin E 
\end{cases}$$
Notice that the entries in the diagonal are all 0 because we do not consider loops. 
And the adjacency matrix is symmetric because we consider undirected graph. 

      
    \end{definition}
    
    
    \begin{proof}
    Suppose there are two zero vectors $\mathbf{0}$ and $\mathbf{0}'$. Then:
    \[
    \mathbf{0} + \mathbf{0}' = \mathbf{0}', \quad \text{by definition of $\mathbf{0}$}
    \]
    But also:
    \[
    \mathbf{0} + \mathbf{0}' = \mathbf{0}, \quad \text{by definition of $\mathbf{0}'$}
    \]
    Hence, $\mathbf{0} = \mathbf{0}'$, proving uniqueness.
    \end{proof}
    
    \begin{remark}
    The zero vector is a specific element in the space, not an absence of element.
    \end{remark}
    
    \begin{conclusion}
    Every vector space has a unique zero vector satisfying $v + 0 = v$ for all $v$.
    \end{conclusion}
\end{document}
    